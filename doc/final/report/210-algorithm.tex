\subsection{Algorithm}
\label{sec:210}

\todorev{Last revised Wed, June 27 at 20:14 by pfac}

The algorithm used by this application is a first order finite volume method. This means that each mesh element only communicates directly with its first level neighbors in the mesh, which makes this a typical case of a stencil computation \cite{mitlab3}. In terms of performance, being a stencil algorithm implies that the operational intensity will most likely remain constant with larger problem sizes \cite{williams2008,parlab2008}. On the other hand, the low order allows for a greater locality of the calculations, and favors parallelization.

The code consists on a preparation stage, where all the required elements are loaded and prepared, and two computation stages, which compose the main loop.

Operations performed in the preparation stage are highly dependent on the implementation being described, as most will require some elements to be properly organized or some values to be previously computed. Common operations, such as loading the necessary data from the described files are constant to every implementation, but may still differ in the structures used to store the data.

A single execution of the two computation stages together form a step in the iterative method behind this application. These stages, also referred in this document as core functions, or kernels, are the \computeflux and \update functions.

In \computeflux, all the edges in the mesh are analyzed, and the flux of pollution to be transfered across that edge is computed, based on the polution level and the velocity vectors of the adjacent cells. A preconfigured value is used as the \dirichlet condition\footnote{The \dirichlet condition is a type of boundary condition used to specify a value taken by the solution in the border of the domain. In the \polu application, this value is constant for the entire mesh border and throughout the execution.}, which replaces the polution level of a second cell for the edges in the border of the mesh.

As for the \update function, it uses the computed flux values to update the polution levels of each cell in the mesh, by adding the individual contribution of each edge of the cell. While triangular cells are prefered, there are no restrictions to the number of edges a cell may have.
