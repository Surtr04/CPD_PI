\subsection{Mesh ordering impact}
\label{subsec:cuda:ordering}

Since GPUs rely heavily on memory organization, it comes as no surprise that the issues stated at \cref{sec:omp:limitations}, and in \cref{fig:locality} become even more relevant.

Memory accesses in a GPU are more efficient when the data being accessed at the same time is contiguous in memory, allowing for a single, coalesced memory request to be issued. The lack of locality provided by \texttt{gmsh} output, as well as the irregular memory access pattern from each kernel difficults this task.

A simple approach was taken to attempt to minimize this problem, which consisted on preprocessing the mesh, moving all cells in the border to the beginning of the structure. The idea was not to increase locality, since the border is only a small subset of the entire mesh, but to attempt more regular accesses in \computeflux, as well as avoiding the divergent branch for border cells. With all border cells being sequential, then only a single warp should diverge on the branch that depends on that.

This solution showed no visible improvements, probably due to the already extremely low kernel execution times, which become bottlenecked by the kernel callback overhead.
