\section{Introduction}

% Contextualização
This document describes an incremental work where the \polu application is studied with the goal of improving its performance. The program uses a Finite Volume method to simulate the spread of a material in a 2D surface.

% Motivação & Objectivos
The goal of this project is to study different approaches to parallelize \polu. Three different parallel implementations are used: the first using shared memory; the second using distributed memory; and a third implementation using a GPU for a massively parallel approach.

% Contextualização (continuação)
A total of four different stages of development contributed for the results shown in this document.
In the first stage, the original implementation was deeply optimized and analyzed to allow parallelism, and two versions were implemented using shared memory and CUDA.
The second stage evolved the shared memory version by further optimizing the sequential version and parallelizing it.
A third stage focused in developing a naive but functional distributed memory implementation, with almost no optimizations.
Based on the results of the previous stages, the final stage was set to focus in improving the initial CUDA version.

In this final document, the previous stages are summarized according to the implementations developed. \Cref{sec:case} describes the case study and the analysis which allowed to identify the parallelism opportunities in the algorithm. The sequential versions implemented, from the original code to the best optimization achieved are discussed in \cref{sec:seq}. \Cref{sec:omp,sec:mpi,sec:cuda} describe the approach, load balance and limitations of the shared memory, distributed memory and massively parallel implementations, respectively.
Final comparative results are shown in \cref{sec:finalresults} and a project conclusion is presented in \cref{sec:conclusion}.

Several details, due to the non-essential nature of the subjects, were pushed to \cref{sec:env,sec:method,sec:roofline}.
These do not include details such as intermediary results obtained in previous stages of the project, which are omitted from this document. Only final speedup values are shown. Please refer to the previous reports for further details about intermediary stages.















% \label{sec:intro}

% This document describes an incremental work where the \polu application, which computes the propagation of a pollutant in a two dimensional environment, was studied in order to find possibilities of optimization and parallelization.

% The \polu application is built on top of the Finite Volume Library (FVL) which is also a focus of study in this document, as a large part of the logic and data structures are implemented on it, rather than on the application itself. In this context, both of them are considered as the whole case study.

% Several changes were performed in the original code, which are fully described in this document. Those changes vary in nature, from simple or low-level code optimization, to higher-level algorithmic changes, in order to allow parallelization and/or improve performance. The data structures used also suffered large changes (originally implemented as \textit{Arrays-of-Pointers}) to \textit{Arrays-of-Structures} at first, and also to \textit{Structure-of-Arrays}. This changes removed excessive dereferencing caused by deep chains of pointers in the original strucutres, effectively reducing memory accesses and improving locality.

% The several phases that composed this project reflect on the multiple approaches and variety of results presented here. In general, the goal is to study the performance impact, advantages and difficulties of different programming paradigms, applied to the \texttt{polu} application. 


% \todo[inline]{O paragrafo aqui a explicar o que é falado no relatorio}

% After the initial analysis of initial sequential code, a shared-memory parallel implementation, using OpenMP \todo{ref sec:omp}, a distributed-memory implementation, with the \textit{Message Passing Interface} (MPI) \cref{sec:mpi} and a GPU implementation using CUDA \cref{sec:cuda} were implemented and profiled.
