\subsubsection{SOA}

\todorev{Last revised on Sat, June 30 at 15:42 by pfac}

While the AOS approach completely removes the need for dereference, using a \textit{Struct-Of-Arrays} (SOA) approach can be more efficient.

One of the problems with the AOS approach is that when the core functions iterate over one of the arrays, cache lines are being filled with the complete structures of these elements.
Yet, neither of the core functions utilizes all the data in that structure, which translates in useless data occupying the cache.
This hurts spatial locality\footnote{If a data element is required, adjacent elements will likely be required in the near future.}, greatly increasing the chances of accessing a mesh element translating into a RAM access.

The SOA approach solves this problem by placing each piece of the elements data in a distinct array. As an example, a single array is created to hold the polution level of all the cells.
The same applies, for example, with the flux of all the edges.

This approach allows the core functions to load only the data required for their computations.
The cache lines are now filled only with useful data, and the pieces which are required by the other functions will only be loaded when necessary.
