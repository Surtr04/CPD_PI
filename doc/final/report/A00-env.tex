\section{Environmental Setup}
\label[appendix]{sec:env}

\todorev{Written on Sun, July 1 at 17:33 by pfac}

The tests described in this document were performed using a very specific subset of nodes from the SeARCH\footnote{\url{http://search.di.uminho.pt}}, here referred as SeARCH Group Hex.

Nodes in SeARCH Group Hex have two hex-core processors (with \intel HyperThreading technology) and 12 to 48 GB of RAM.Further detail regarding the hardware of these nodes can be found in \cref{tab:grouphex}.

These nodes were used for every test performed in this document.
Sequential, shared memory and GPU tests relied used only one node, where tests for distributed memory used two distinct nodes. Inside this group, tests were only limited to the subset with the highest processor clock frequency, not making any distiction between the nodes.
The only exceptions are the GPU tests which used a very specific node with a Tesla M2070 GPU computing module.
Further detail regarding this GPU module can be found in \cref{tab:tesla:m2070}.

\begin{table}[!htp]
	\begin{center}
		\begin{tabular}{lc}
			\hline
			Processors per node: & 2	\\
			Processor model: & \intel\xeon X5650\\
			Cores per processor: & 6	\\
			Threads per core: & 2	\\
			Clock frequency: & 2.66 GHz	\\
			\hline
			L1 cache: & 32 KB + 32 KB per core	\\
			L2 cache: & 256 KB per core	\\
			L3 cache: & 12 MB shared	\\
			RAM: & 12 to 48 GB	\\
			\hline
		\end{tabular}
		\caption[SeARCH Group Hex hardware description]{SeARCH Group Hex hardware description. See \cite{xeon5600} for further detail about this processor.}
		\label{tab:grouphex}
	\end{center}
\end{table}

\begin{table}[!htp]
	\begin{center}
		\begin{tabular}{lc}
			\hline
			GPUs per node: & 1	\\
			Model: & \tesla M2070\\
			CUDA cores: & 448	\\
			Peak (double): & 515 Gflops	\\
			\hline
			Dedicated Memory: & 6 GB	\\
			Bandwidth: & 148 GB/s\\
			\hline
		\end{tabular}
		\caption[SeARCH Group Hex GPU computing module hardware description]{SeARCH Group Hex GPU computing module hardware description. See \cite{teslaM2070} for further detail about this GPU.}
		\label{tab:tesla:m2070}
	\end{center}
\end{table}

Results obtained in a second type of nodes are shown in this document only to present a roofline.
While this roofline should refer to the SeARCH Group Hex, this being the environment used for comparison, in previous stage of this project the \texttt{PAPI} library could only be used in a very specific node of SeARCH Group 201.
Build the roofline for the Group Hex nodes, which meant rerunning tests with the \texttt{PAPI} library, was not possible in this stage due to time constrictions.
Attempts were made in previous project stages, but the STREAM benchmark\footnote{\url{http://www.cs.virginia.edu/stream/}} returned less memory bandwidth than in Group 201, which is not true.
Later analysis showed the problem was in the fact that such benchmark must not be run in Group Hex nodes using all the supported parallelism (6 threads, instead of 24, achieved around 15 GB/s).

SeARCH Group 201 is composed by nodes with two dual-core processors and 4 GB of RAM. Further detail regarding these nodes can be found in \cref{tab:group201}.

\begin{table}[!htp]
	\begin{tabular}{ll}
		\hline
		Processors per node: & 2	\\
		Processor model: & Intel\textregistered Xeon\textregistered E5130\\
		Cores per processor: & 2	\\
		Threads per core: & 1	\\
		Clock frequency: & 2.00 GHz	\\
		\hline
		L1 cache: & 32 KB + 32 KB per core	\\
		L2 cache: & 4 MB shared	\\
		L3 cache: & N/A	\\
		RAM: & 4 GB	\\
		\hline
	\end{tabular}
	\caption[SeARCH Group 201 hardware description]{SeARCH Group 201 hardware description. See \cite{xeon5100} for further detail about this processor.}
	\label{tab:group201}
\end{table}