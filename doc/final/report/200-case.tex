\section{Case Study}
\label{sec:case}

\todorev{Last revised Wed, June 27 at 20:12 by pfac}

The application analyzed in this document, here called \polu, computes the spread of a material (e.g. a pollutant) in a bidimensional surface through the course of time. This surface is discretely represented as a mesh, composed mainly of edges and cells. The input is given in the form of XML files, passed as arguments:
\begin{itemize}
	\item Mesh description;
	\item Velocity vector for each cell;
	\item Initial pollution values of each cell.
\end{itemize}

Both the input and the output of the program can be converted to the \texttt{msh} format, compatible with the \texttt{gmsh} mesh generator, for data visualization. This is done using conversion programs, also written using the FVL library, which handle the conversion between the XML schema used by the library, and the mesh format specification of \texttt{gmsh} \cite{gmsh}.

\subsection{Algorithm}

The algorithm used by the \polu application is a first order finite volume method. This means that each mesh element only communicates directly with its first level neighbors in the mesh, which makes this a typical case of a stencil computation. In terms of performance, being a stencil algorithm implies that the operational intensity \todo[inline]{ref ao paper do roofline} will most likely remain constant with larger problem sizes. On the other hand, the low order allows for a greater locality of the calculations, and favors parallelization.

The code consists on a preparation stage, where all the required elements are loaded and prepared, and two computation stages, which compose the main loop.

Operations performed in the preparation stage are highly dependent on the implementation being described, as most will require some elements to be properly organized or some values to be previously computed. Common operations, such as loading the necessary data from the described files are constant to every implementation, but may still differ in the structures used to store the data.

A single execution of the two computation stages together form a step in the iterative method behind this application. These stages, also referred in this document as core functions, or kernels, are the \computeflux and \update functions.

In \computeflux, all the edges in the mesh are analyzed, and the flux of pollution to be transfered across that edge is computed, based on the polution level and the velocity vectors of the cells it connects. A preconfigured value is used as the \dirichlet condition\footnote{The \dirichlet condition is a type of boundary condition used to specify a value taken by the solution in the border of the domain. In the \polu application, this value is constant throughout the execution}, which replaces the polution level of a second cell for the edges in the border of the mesh.

As for the \update function, it uses the computed flux values to update the polution levels of each cell in the mesh, by adding the individual contribution of each edge of the cell. While triangular cells are prefered, there are no restrictions to the number of edges a cell may have.

\subsection{Parallelism Opportunities}
\label{sec:220}

\todorev{Last revised and changed on Wed, June 27 at 20:14 by pfac}

A typical approach to parallelism in stencil algorithms is based on the premise that a given point of the domain can be computed once the surrounding points (its dependencies) are ready.
Due to this being a first order algorithm, these dependencies are extremely local (computations for a cell depend only on the immediately adjacent edges, and vice-versa).

Theoretically, this should the basis for the parallelism strategy: mesh subsets can be computed once their directly connected elements are up-to-date, even though other unrelated mesh subsets would possibly be in an older state.
However, such an approach also adds a higher degree of complexity, and involves deep changes to the underlying algorithm.
As the purpose of this document is not to study a better alternative algorithm to solve the problem \polu is aimed at, this approach was not followed.

Instead of the typical strategy, each core function can be analyzed as an independent task. During \polu's main loop both \computeflux and \update depend on each other to perform their tasks. \update requires flux from all edges to be previously computed in \computeflux, which in turn requires that all pollution values are up-to-date to correctly compute the flux for the next iteration. This creates two implicit synchronization points in the main loop and is a consequence of the heartbeat characteristics of the problem. Yet, both functions may be completely executed in parallel, performing the required computations for each internally iterated mesh element.

