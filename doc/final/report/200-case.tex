\section{Case Study}

The application analyzed in this document, called here \polu, computes the flux of a material (e.g. a pollutant) through a bidimensional surface. This surface is described as a mesh, composed mainly of edges and cells. The input is given in the form of XML files, and in total consists of:

\begin{itemize}
	\item Mesh description;
	\item Velocity vector for each cell;
	\item Initial pollution values of each cell.
\end{itemize}

Both the input and the output of the program can be converted to the \texttt{msh} format, compatible with the \texttt{gmsh} mesh generator, for data visualization. This is done using conversion programs, also written using the FVL library, which handle the conversion between the XML schema used by the library, and the mesh format specification of \texttt{gmsh}\cite{gmsh}.

\subsection{Algorithm}

The algorithm used by the \polu application is a first order finite volume method. This means that each mesh element only communicates directly with its first level neighbors in the mesh, which makes this a typical case of a stencil computation. In terms of performance, being a stencil algorithm implies that the operational intensity \todo[inline]{ref ao paper do roofline} will most likely remain constant with larger problem sizes. On the other hand, the low order allows for a greater locality of the calculations, and favors parallelization.

The code consists on a preparation stage, where all the required elements are loaded and prepared, and two computation stages, which compose the main loop.

Operations performed in the preparation stage are highly dependent on the implementation being described, as most will require some elements to be properly organized or some values to be previously computed. Common operations, such as loading the necessary data from the described files are constant to every implementation, but may still differ in the structures used to store the data.

A single execution of the two computation stages together form a step in the iterative method behind this application. These stages, also referred in this document as core functions, or kernels, are the \computeflux and \update functions.

In \computeflux, all the edges in the mesh are analyzed, and the flux of pollution to be transfered across that edge is computed, based on the polution level and the velocity vectors of the cells it connects. A preconfigured value is used as the \dirichlet condition\footnote{The \dirichlet condition is a type of boundary condition used to specify a value taken by the solution in the border of the domain. In the \polu application, this value is constant throughout the execution}, which replaces the polution level of a second cell for the edges in the border of the mesh.

As for the \update function, it uses the computed flux values to update the polution levels of each cell in the mesh, by adding the individual contribution of each edge of the cell. While triangular cells are prefered, there are no restrictions to the number of edges a cell may have.

\subsection{Parallelism Oportunities}
\todo[inline]{Explain what can be executed in parallel}
\todo[inline]{Mention this is a heartbeat}

During the main loop of this program both core functions, \computeflux and \update, depend on each other to perform their tasks. \update requires flux from all edges to be previously computed in \computeflux, which in turn requires that all pollution values are up-to-date to correctly compute the flux for the next iteration. This creates two implicit synchronization points in the main loop, and is a consequence of the heartbeat characteristics of the problem.

This allows both functions to be looked at as individual tasks, that may be subject to different parallelization approaches. Both functions perform calculations using the entire mesh, but they differ in the element used: while \computeflux iterates over edges, \update iterates over cells.

\todo[inline,color=green!40]{Ainda tentei acabar eu isto mas não sei bem onde querias ir com este texto, por isso acaba tu. Pode ser útil um parágrafo a explicar que nos algoritmos de stencil é comum deixar-se que diferentes partes da mesh estejam em diferentes iterações por só terem dependência local, mas que no nosso caso por ser um heartbeat isto não é de todo trivial de implementar.}
