\section{Second-order schemes}
\label{sec:400}

Second-order scheme are achieved using the reconstruction process equipped with a limiting procedure
to ensure the stability of the solution. For two-dimensional domain, the generic scheme writes
$$
\phi_i^{n+1} = \phi_i^n - \Delta{t} \sum_{j \in \nu(i)} {|e_{ij}| \over |K_i|} 
F(\phi_{ij}^n, \phi_{ji}^n, n_{ij}) 
$$
where $\phi_{ij}^n$ and $\phi_{ji}^n$ represent more accurate representation of $\phi$ on both
side of the interface $e_{ij}$.\\
Unfortunately, the choice $(a_i,b_i)=(\widehat a_i,\widehat b_i)$ is only correct when dealing
with smooth solution but in the vicinity of a discontinuity, the Gibbs phenomena occurs
and generate spurious solutions. To this end a nonlinear procedure is mandatory to guarantee
the stability of the method. 
Two different strategies can be considered : a {\it a priori} strategy (MUSCL method)
where the limiting procedure is performed before the the update; or a {\it a posteriori} strategy 
(MOOD method) where a candidate solution is computed and modified to provide the stability.

For one-dimensional problem, the scheme is very similar and writes
$$
\phi_i^{n+1}=\phi_i^{n}-\frac{\Delta t}{\Delta x}\Big \{
F(\phi^{n,-}_{i+1/2},\phi^{n,+}_{i+1/2})-F(\phi^{n,-}_{i-1/2},\phi^{n,+}_{i-1/2} \Big \}
$$
where $\phi^{n,-}_{i+1/2}$ and $\phi^{n,+}_{i+1/2}$ represent more accurate approximations
on both side of interface $x_{i+1/2}$.
As mention in the previous section, high-order in time is guarantee with the RK2 scheme both for the
one- and the two-dimensional case.
\subsection{MUSCL method}
\label{sec:410}
The MUSCL method is based on a local limitation of the slope to provide stability.
Actually, maximum principle is the main criterion and write
\begin{equation}
\min_{j\in \nu(i)}(\phi^n_i,\phi_j)\leq u^{n+1}_i \leq \max_{j\in \nu(i)}(\phi^n_i,\phi_j).
\label{eq:MP_criterion}
\end{equation}
To achieve such a property, we multiply the reconstructed slope $\widehat \sigma_i$ by a
coefficient $\chi_i\in[0,1]$ such that with the new slope $\sigma_i=\chi_i\widehat \sigma_i$
the update solution $\phi^{n+1}_i$ satisfies the restriction (\ref{eq:MP_criterion}).\\
There exists a lot of method to determine the limiter but we propose the one introduce
by Barth and Jepherson. To this end, let denote 
$$
\widehat \phi_{ij}=\phi_i+\widehat \sigma_i B_iM_{ij}
$$
where $M_{ij}$ is the midpoint of edge $e_{ij}$. In a first step, 
for any cell $K_i$ and $j\in\nu(i)$, we evaluate the quantity
$$
\chi_{ij} = \left\{\begin{array}{l l}
\frac{\max(\phi_j,\phi_i)-\phi_i}{\phi_{ij}-\phi_i} & \textrm{ if } \phi_{ij}> \max(\phi_j,\phi_i) \\
\frac{\min(\phi_j,\phi_i)-\phi_i}{\phi_{ij}-\phi_i}  &\textrm{ if } \phi_{ij} <\min(\phi_j,\phi_i) \\
1 & \text{otherwise}
\end{array}\right.
$$
Then we define 
$$
\chi_i=\min_{j\in\nu(i)} \chi_{ij}
$$
and the reconstructed values by
$$
u_{ij} = u_i + \chi_i\widehat \sigma_i  B_i M_{ij}
$$


For the one-dimensional situation, we use the limiter operator such as the minmod function
setting
$$
\chi_i=(\phi_{i+1}-\phi_{i})\text{minmod}\left (\frac{\phi_{i}-\phi_{i-1}}{\phi_{i+1}-\phi_{i}}\right ) 
$$
with minmod$(\alpha)=0$ if $\alpha<0$;  minmod$(\alpha)=1$ if $\alpha>1$ and minmod$(\alpha)=\alpha$
otherwise.
We then compute the reconstructed values with
$$
\phi_{i-1/2}^{n,+}=\phi_i^n-\chi_i^n \widehat \sigma_i^n \frac{\Delta x}{2},\quad
\phi_{i+1/2}^{n,-}=\phi_i^n+\chi_i^n \widehat \sigma_i^n \frac{\Delta x}{2}
$$
The algorithm for a one forward Euler step is
\begin{enumerate}
\item mean values $(\phi^n_i)$ are known.
\item compute the slope $\sigma_i$
\item compute the limiter $\chi_i$
\item compute the reconstructed values
\item compute the flux
\item time update 
\end{enumerate}
One has to perform two time the procedure when using the RK2 technique to provide
second-order in time.\\
{\it Remark.} The scheme is {\it a priori} because the limiting procedure has been performed before
the update.  
 
\subsection{MOOD Scheme}
\label{sec:420}

Multi-dimensional Optimal Order Detection (or MOOD) method, operates on a \textit{a posteriori} approach, as oposed to the \textit{a priori} approach from more classical methods like MUSCL.
An initial, unlimited polynomial reconstruction is calculated, building a candidate solution $u^{\star}$. Each cell of the domain is initialized with a reconstruction of a higher polynomial degree, in this case $d=1$.

The solution is then checked for problems by a detector function. The detector checks the following condition for every cell:

$$ \min_{j \in \underline{v}(i)}(u_i^{\star}, u_j) \le u_i^{\star} \le \max_{j \in \underline{v}(i)}(u_i^{\star}, u_j) $$

 
When a cell does not meet the required condition it is considered invalid. The polynomial degree for the reconstruction on that cell is decreased, which in this case corresponds to $d=0$, falling to the first-order scheme for that point of the domain.
 hen the entire candidate solution has no errors detected, the time step is complete, with

$$ u = u^{\star} $$

The initial steps of the MOOD implementation are equivalent to the previous version, since they are independent of the limitation strategy. However, the philosophy of both approaches is different. While MUSCL, being an \textit{a priori} method, attempts to predict errors and add a limiter to the polynomial reconstruction, preventing them from happening, MOOD lets those errors happen, and then looks for them, rebuilding the reconstruction for the problematic points.

From an implementational point of view, this is also a very different strategy, since it involves an inner loop to iterate over the candidate solution, until no more errors are detected. Also, since the problematic points can, and usually will be only a small percentage of the entire domain, it might become more difficult to achieve an efficient parallelization strategy for this method, when compared to the straightforward approach on the previous method.