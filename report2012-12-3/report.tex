\documentclass[abstract=on,10pt,twocolumn]{scrartcl}

\usepackage[utf8]{inputenc}
\usepackage{todonotes}

\usepackage[paper=a4paper,top=1.5cm,left=1.9cm,right=1.9cm]{geometry}
\usepackage[colorlinks=true,linkcolor=black,citecolor=black]{hyperref}
\usepackage{cleveref}
\usepackage{relsize}% relative font sizes
\usepackage{amsmath,amssymb}
\usepackage{algorithm}
\usepackage{algpseudocode}
\usepackage{epstopdf}
\usepackage{graphicx}
\usepackage{caption}
\usepackage{subcaption}
\usepackage{indentfirst}
\usepackage{datetime}
\usepackage[retainorgcmds]{IEEEtrantools}

% declare the path(s) where your graphic files are
\graphicspath{{images/}}

\usepackage{xspace}

% Add "Appendix" to the appendices titles, but not to the references
\usepackage{ifthen}
\newcommand*{\appendixmore}{%
  \renewcommand*{\othersectionlevelsformat}[1]{%
    \ifthenelse{\equal{##1}{section}}{\appendixname~}{}%
    \csname the##1\endcsname\autodot\enskip}
  \renewcommand*{\sectionmarkformat}{%
    \appendixname~\thesection\autodot\enskip}
}

%%%%%%%%%%%%%%%%
%  title page  %
%%%%%%%%%%%%%%%%
\titlehead{University of Minho \\Department of Mathematics}

\title{GPU implementation of finite volume schemes}

\author{
    \\Miguel Palhas\\
      \texttt{\smaller pg19808@alunos.uminho.pt}
  \and
  \\Stéphane~Clain (co-Advisor)\\
    \texttt{\smaller clain@math.uminho.pt}
}

\newdateformat{mmmyyyydate}{\monthname[\THEMONTH] \THEYEAR}
\newcommand{\docdate}{\mmmyyyydate\today}
\date{Braga, \docdate}

\newcommand{\todosec}{\todo[inline,color=black!20]}

\newcommand{\sumneighbors}{\ensuremath{\sum_{j \in \underline{v}(i)}}}
\newcommand{\matA}{\ensuremath{\left( \begin{array}{cc} A_{11} & A_{12} \\ A_{21} & A_{22} \\ \end{array} \right)}}
\newcommand{\vecGrad}{\ensuremath{\left( \begin{array}{c} b \\ c \\ \end{array} \right)}}
\newcommand{\vecR}{\ensuremath{\left( \begin{array}{c} r_1 \\ r_2 \\ \end{array} \right)}}

%
% DOCUMENT
%

\begin{document}

\maketitle

%
% ABSTRACT
%
% for Computer Society papers, we must declare the abstract and index terms
% PRIOR to the title within the \IEEEcompsoctitleabstractindextext IEEEtran
% command as these need to go into the title area created by \maketitle.
\IEEEcompsoctitleabstractindextext{%
\begin{abstract}
%\boldmath
\todo[inline]{This template is quite acceptable for a paper, but IMO it is unfit for a technical/scientific report for a class. SCREW YOU. A tua prima!}


\end{abstract}
% IEEEtran.cls defaults to using nonbold math in the Abstract.
% This preserves the distinction between vectors and scalars. However,
% if the journal you are submitting to favors bold math in the abstract,
% then you can use LaTeX's standard command \boldmath at the very start
% of the abstract to achieve this. Many IEEE journals frown on math
% in the abstract anyway. In particular, the Computer Society does
% not want either math or citations to appear in the abstract.

% Note that keywords are not normally used for peerreview papers.
\begin{IEEEkeywords}
Integrated Project, Finite Volume, mesh, C/C++, OpenMP, MPI, CUDA
\end{IEEEkeywords}
}
\section{Introduction}
\label{sec:intro}

This document describes an incremental work where the \texttt{polu} application, which computes the propagation of a pollutant in a two dimensional environment, was studied in order to find possibilities of optimization and/or parallelization.

The \texttt{polu} application is built on top of the Finite Volume Library (FVL) which is also a focus of study in this document, as a large part of the logic and data structures are implemented on it, rather than on the application itself. In this context, both of them are considered as a whole case study.

Several changes were performed in the original code, which are fully described in this document. Those changes vary in nature, from simple or low-level code optimization, to higher-level algorithmic changes, in order to allow parallelization and/or improve performance. The data structures used also suffered large changes (originally implemented as \textit{Arrays-of-Pointers}) to \textit{Arrays-of-Structures} at first, and also to \textit{Structure-of-Arrays}. This changes removed excessive dereferencing caused by deep chains of pointers in the original strucutres, effectively reducing memory accesses and improving locality.

The several phases that composed this project reflect on the multiple approaches and variety of results presented here. In general, the goal is to study the performance impact, advantages and difficulties of different programming paradigms, applied to the \texttt{polu} application. 


\todo[inline]{O paragrafo aqui a explicar o que é falado no relatorio}

After the initial analysis of initial sequential code, a shared-memory parallel implementation, using OpenMP \todo{ref sec:omp}, a distributed-memory implementation, with the \textit{Message Passing Interface} (MPI) \todo{ref sec:mpi}, and a GPU implementation (using CUDA) \todo{ref sec:cuda} were implemented and profiled.

\section{First Order Scheme}
\label{sec:200}

The initial approach consisted on a first order scheme that solves the problem.

$$ \int_{K_i} \partial{t} u + dir(V_{n_{ij}})u = 0 $$

Let $K_i$ and $K_j$ be two neighbor cells of the system, and $e_{ij}$ the interface between them. In a first step, the method computes the flux passing from $K_i$ to $K_j$, based on the current average values for each cell, and the velocity field $\vec{V}$. The flux is computed for each edge using the immediate neighbor information. The first-order finite volume scheme used is as follows:

$$ u_i^{n+1} = u_i^n - \Delta{t} \sum_{j \in \underline{v}(i)} {|e_{ij}| \over |K_j|} F(u_i^n, u_j^n, n_{ij}) $$

Where the flux calculation for each interface is:

$$ F(u_i^n, u_j^n, n_{ij}) = [V_{n_{ij}}]^{+}u_i^n + [V_{n_{ij}}]^{-}u_j^n $$

The implemented method works on a bidimensional mesh based on some assumptions. One of the assumptions is that edges should always have at least on adjacent cell, and that cell will be refered to as the left cell. The other cell, called the right cell, may not exist, which is the case for the edges on the border of the domain. On those cases, it is assumed:

$$ u_j = dc $$

Where $dc$ is the \textit{Dirichlet Boundary Condition}. In this particular work, this value was kept constant throughout the computation.

First other schemes like these are very popular due to its simplicity, but suffer from large amounts of numerical diffusion, which leads to poor accuracy and the smoothing of discontinuities as time passes. However, this approach was useful to provide something to work on top of, when targeting the more accurate approaches with the second-order schemes.
\section{Polynomial Reconstruction}
\label{sec:300}
Polynomial reconstruction is an important building-block to achieve high-order accuracy. We just present the situation for a one-degree reconstruction for the sake of simplicity. Let us consider
a generic cell $K_i$, assume that me known an approximation of the mean values $\phi_i$ and
$\phi_j$, $j\in\nu(i)$. We seek a linear function of type
$$ 
\widehat{\phi}_i(X) = \phi_i + \sigma(B_iX)=
\phi_i + \left( \begin{array}{c}a_i \\ b_i \end{array}\right) (B_iX)
$$
where $B_i$ stand for the centroid of cell $K_i$. To fix vector $\sigma_i=(a_i,b_i)$ 
we introduce the functional 
$$
E(a_i,b_i) = \sum_{j \in \underline{v}(i)} [\widehat{\phi}_i(B_j) - \phi_j]^2 
$$
and we state that the coefficients $\widehat a_i,\widehat b_i$ of vector $\widehat \sigma_i$ 
are those which corresponds to the minimum of 
functional $E(a_i,b_i)$.
This provide a least square problem we cast in the matricial form
\begin{eqnarray*}
& &\sum_{j\in\nu(i)} \left [\begin{array}{cc}
(B_iB_j)^2_x &  (B_iB_j)_x(B_iB_j)_y \\
(B_iB_j)_x(B_iB_j)_y &\ (B_iB_j)^2_y 
\end{array}\right ]
\left( \begin{array}{c}
\widehat a_i \\ \widehat b_i 
\end{array}\right)\\
& &=
\sum_{j\in\nu(i)} \left( \begin{array}{c}
(\phi_j-\phi_i)(B_iB_j)_x\\
 (\phi_j-\phi_i)(B_iB_j)_y
\end{array}\right).
\end{eqnarray*}
Note that the approximation is exact when  function $\phi$ is linear.

For the one-dimensional case, we slightly modify the reconstruction process introducing several 
type of slope. Since we only deal with periodic condition the fictitious cell $I+1$ corresponds
to cell $1$ while the fictitious cell $0$ corresponds to cell $I$. Assume that mean values
$\phi_i$, $i=1,...,N$ are known, the we consider following the slopes
$$
\widehat \sigma_{i+1/2}=\frac{\phi_{i+1}-\phi_i}{\Delta x}, \quad 
\widehat \sigma_i=\frac{\phi_{i+1}-\phi_{i-1}}{2\Delta x}
$$
Now for a given slope $\sigma_i$, we introduce the local linear reconstruction as
$$
\widehat \phi_i(x)=\phi_i+\sigma_i(x-c_i)
$$
where the slope $\sigma_i$ will be determine in function of  $\widehat \sigma_{i-1/2}$,
$\widehat \sigma_{i}$ and $\widehat \sigma_{i+1/2}$.
Note that the reconstruction is exact when $\phi$ is a linear function.

\section{Second Order Schemes}
\label{sec:400}

\todosec{Introduzir os esquemas de 2ª ordem}

For a second order scheme to implemented, instead of using the average values for each cell to compute the flux through an interface, the polynomial reconstruction is used to estimate the average value along the edge. We call $u_{ij}$ to the estimation of the edge value using the reconstruction from cell $i$, and $u_{ji}$ to the value using the reconstruction from cell $j$. The new flux equation is given by:

$$
F(u_{ij}, u_{ji}, n_{ij}) = [V_{n_{ij}}]^{+}u_{ij} + [V_{n_{ij}}]^{-}u_{ji}
$$

Two different versions were developed of a second order scheme, different mostly on the limitation strategy used to limit the reconstruction, and lead to more accurate solutions in cases of slopes or sharp discontinuities.

\subsection{MUSCL method}
\label{sec:410}
The MUSCL method is based on a local limitation of the slope to provide stability.
Actually, maximum principle is the main criterion and write
\begin{equation}
\min_{j\in \nu(i)}(\phi^n_i,\phi_j)\leq u^{n+1}_i \leq \max_{j\in \nu(i)}(\phi^n_i,\phi_j).
\label{eq:MP_criterion}
\end{equation}
To achieve such a property, we multiply the reconstructed slope $\widehat \sigma_i$ by a
coefficient $\chi_i\in[0,1]$ such that with the new slope $\sigma_i=\chi_i\widehat \sigma_i$
the update solution $\phi^{n+1}_i$ satisfies the restriction (\ref{eq:MP_criterion}).\\
There exists a lot of method to determine the limiter but we propose the one introduce
by Barth and Jepherson. To this end, let denote 
$$
\widehat \phi_{ij}=\phi_i+\widehat \sigma_i B_iM_{ij}
$$
where $M_{ij}$ is the midpoint of edge $e_{ij}$. In a first step, 
for any cell $K_i$ and $j\in\nu(i)$, we evaluate the quantity
$$
\chi_{ij} = \left\{\begin{array}{l l}
\frac{\max(\phi_j,\phi_i)-\phi_i}{\phi_{ij}-\phi_i} & \textrm{ if } \phi_{ij}> \max(\phi_j,\phi_i) \\
\frac{\min(\phi_j,\phi_i)-\phi_i}{\phi_{ij}-\phi_i}  &\textrm{ if } \phi_{ij} <\min(\phi_j,\phi_i) \\
1 & \text{otherwise}
\end{array}\right.
$$
Then we define 
$$
\chi_i=\min_{j\in\nu(i)} \chi_{ij}
$$
and the reconstructed values by
$$
u_{ij} = u_i + \chi_i\widehat \sigma_i  B_i M_{ij}
$$


For the one-dimensional situation, we use the limiter operator such as the minmod function
setting
$$
\chi_i=(\phi_{i+1}-\phi_{i})\text{minmod}\left (\frac{\phi_{i}-\phi_{i-1}}{\phi_{i+1}-\phi_{i}}\right ) 
$$
with minmod$(\alpha)=0$ if $\alpha<0$;  minmod$(\alpha)=1$ if $\alpha>1$ and minmod$(\alpha)=\alpha$
otherwise.
We then compute the reconstructed values with
$$
\phi_{i-1/2}^{n,+}=\phi_i^n-\chi_i^n \widehat \sigma_i^n \frac{\Delta x}{2},\quad
\phi_{i+1/2}^{n,-}=\phi_i^n+\chi_i^n \widehat \sigma_i^n \frac{\Delta x}{2}
$$
The algorithm for a one forward Euler step is
\begin{enumerate}
\item mean values $(\phi^n_i)$ are known.
\item compute the slope $\sigma_i$
\item compute the limiter $\chi_i$
\item compute the reconstructed values
\item compute the flux
\item time update 
\end{enumerate}
One has to perform two time the procedure when using the RK2 technique to provide
second-order in time.\\
{\it Remark.} The scheme is {\it a priori} because the limiting procedure has been performed before
the update.  
 
\subsection{MOOD Scheme}
\label{sec:420}

Multi-dimensional Optimal Order Detection (or MOOD) method, operates on a \textit{a posteriori} approach, as oposed to the \textit{a priori} approach from more classical methods like MUSCL.
An initial, unlimited polynomial reconstruction is calculated, building a candidate solution $u^{\star}$. Each cell of the domain is initialized with a reconstruction of a higher polynomial degree, in this case $d=1$.

The solution is then checked for problems by a detector function. The detector checks the following condition for every cell:

$$ \min_{j \in \underline{v}(i)}(u_i^{\star}, u_j) \le u_i^{\star} \le \max_{j \in \underline{v}(i)}(u_i^{\star}, u_j) $$

 
When a cell does not meet the required condition it is considered invalid. The polynomial degree for the reconstruction on that cell is decreased, which in this case corresponds to $d=0$, falling to the first-order scheme for that point of the domain.
 hen the entire candidate solution has no errors detected, the time step is complete, with

$$ u = u^{\star} $$

The initial steps of the MOOD implementation are equivalent to the previous version, since they are independent of the limitation strategy. However, the philosophy of both approaches is different. While MUSCL, being an \textit{a priori} method, attempts to predict errors and add a limiter to the polynomial reconstruction, preventing them from happening, MOOD lets those errors happen, and then looks for them, rebuilding the reconstruction for the problematic points.

From an implementational point of view, this is also a very different strategy, since it involves an inner loop to iterate over the candidate solution, until no more errors are detected. Also, since the problematic points can, and usually will be only a small percentage of the entire domain, it might become more difficult to achieve an efficient parallelization strategy for this method, when compared to the straightforward approach on the previous method.
\section{Numerical Results}
\label{sec:500}
We tested the two methods in the context of the one-dimnesional geometry where we deal with two representative situations:
the transport of a very smooth function  and a discontinuous function. We focus the analyse on two particular points:
the accuracy of the methods for smooth solution and the stability of the methods for rough solution.
We assume that the velocity is $u=1$ and prescribe periodic boundary condition. The mesh is a uniform $I+1$ points
subdivision with space parameter $\Delta x=\frac{1}{I}$. Time step $\Delta t$ is controled by the CFL condition and we 
shall set $\Delta t=0.6 \Delta x$.
In the sequel, $S1$ is the first-order scheme while $S2MUCL$ and $S2MOOM$ are the second-order scheme with the MUSCL and 
the MOOD limiting strategies respectively.
\subsection{The smooth case}
The initial solution is the simple sine function $\phi^0(x)=\sin(2\pi x)$ we transport at velocity $u=1$.
We plot in figure (\ref{fig_sine}) the exact solution and the approximations using he the MUSCL and the MOOD method
after a complete revolution. Clearly, the MOOD method provides the best approximation and manage to reduce
the diffusiove effect of the FV scheme. We report in figure (\ref{fig_conv1}-\ref{fig_convinfty}) the $L^1$ and $L^\infty$
convergence curve and we get an effective second-order convergence. We remark that the MOOD technique gives a smaller
error but the convergence rates are indentical in the two cases.
We do not provide the table convergence data since the curves are really straighforward. 
\begin{figure}[ht]
\begin{center}
\begin{tabular}{c}
\includegraphics[width=7cm,height=5cm, clip=true,viewport=80 200 600 580
]{figure_sin.pdf}
\end{tabular}
\end{center}
\caption{\label{fig_sine} \footnotesize Numerical solutions at $t=1.0$ s for the sine function: 
exact solution in red, muscl method in green, mood method in blue.}
\end{figure}


\begin{figure}[t]
\begin{center}
\begin{tabular}{c}
\includegraphics[width=7cm,height=4cm, clip=true,viewport=80 200 600 580
]{curve1.pdf}
\end{tabular}
\end{center}
\caption{\label{fig_conv1}  \footnotesize $L^1$ error convergen using the log-log scale: muscl method in green, mood method in blue.
We observe an effective second-order convergence.}
\end{figure}

\begin{figure}[t]
\begin{center}
\begin{tabular}{c}
\includegraphics[width=7cm,height=4cm, clip=true,viewport=80 200 600 580
]{curve0.pdf}
\end{tabular}
\end{center}
\caption{\label{fig_convinfty}  \footnotesize $L^\infty$ error convergen using the log-log scale: 
muscl method in green, mood method in blue.
We observe an effective second-order convergence.}
\end{figure}

\subsection{The rough case}
The initial solution is the Heaviside function centered at $1/2$ given by $\phi^0(x)=H(x-1/2)$ 
we transport at velocity $u=1$.
We plot in figure (\ref{fig_heavi}) the exact solution and the approximations using he the MUSCL and the MOOD method
after a complete revolution. One more time, the MOOD method provides the best approximation and manage to reduce
the diffusiove effect of the FV scheme in the vicinity of the discontinuity.
We do not provide any convergence curve since we deal with a discontinuous function. Nevertheless, we have check a 
half-order convergence in norm $L^1$ which characterize the convergence rate for a discontinuous solution.
\begin{figure}[ht]
\begin{center}
\begin{tabular}{c}
\includegraphics[width=7cm,height=5cm, clip=true,viewport=80 200 600 580
]{figure_heavi.pdf}
\end{tabular}
\end{center}
\caption{\label{fig_heavi} Numerical solutions at $t=1.0$ s for the step function: 
exact solution in red, muscl method in green, mood method in blue.}
\end{figure}





\section{CUDA Implementation}
\label{sec:600}

\todosec{CUDA implementation}
Part of the work of this project was to study the parallelization methods and issues of finite volume schemes in a massively parallel architecture, specifically a GPU, using CUDA as the development technology. Some considerations can be made about the parallelization of each of the schemes presented.

The first-order scheme is the simpler one, with each value of the solution at the end of each iteration being computed based only on the strictly adjacent elements of the mesh. As a result, it should come as no surprise that it was also the one that showed better performance. The higher locality gained from only using adjacent values for each edge allows for much greater locality and less memory overhead than the second-order counterparts.

As for the second-order schemes, before any actual measurements were done, it was expected that the MUSCL would provide a greater occupancy of the GPU, which does not necessarily mean its execution time would be better. The MOOD method spends a significant portion of its time fixing errors on the detected cells. Since these cells usually represent a very small percentage of the entire domain, this means that most of the computational units will be idle, or doing useless work. In MUSCL, this is not the case, as for every interface there is a strict process of building the initial reconstruction, computing the $\Phi$ limiter for that interface, and then limit the reconstruction.
\section{Conclusions}
\label{sec:900}

This report presented a summary of the work done on two second-order schemes. The first one, using a more classic \textit{a priori} approach, was based on a MUSCL scheme to limite the polynomial reconstruction. A later method, with the \textit{a posterior} MOOD scheme, focused on building an initial unlimited reconstruction, and iteratively reducing the polynomial degree of the problematic points that were later detected with that reconstruction.

\todosec{Conclusions: efficiency cpu vs gpu}

%
% BIBLIOGRAPHY
%
%\bibliographystyle{IEEEtran}
%\bibliography{../bib/strings,../bib/articles,../bib/inproceedings,../bib/manuals,../bib/misc,../bib/techreports}

% \printbibliography

%
% APPENDIXES
%

\newpage
\appendix

\end{document}
The case study and application of this method is to compute the distribution of a pollutant in a surface, and its distribution as time passes.
$$V = \left( \begin{array}{c} 1 \\ 0 \end{array} \right) $$
applied to a bidimensional domain. The domain is represented as a 2D mesh, composed of cells connected by their interfaces, and contains a velocity field $\vec{V}(x, y)$ to control the direction of propagation. This velocity field is kept constant throughout the entire computation. For simplification purposes, it has only one direction: